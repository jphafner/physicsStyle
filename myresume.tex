%%%%%%%%%%%%%%%%%%%%%%%%%%%%%%%%%%%%%%%%%
% Twenty Seconds Resume/CV
% LaTeX Template
% Version 1.0 (14/7/16)
%
% Original author:
% Carmine Spagnuolo (cspagnuolo@unisa.it) with major modifications by 
% Vel (vel@LaTeXTemplates.com) and Harsh (harsh.gadgil@gmail.com)
%
% License:
% The MIT License (see included LICENSE file)
%
%%%%%%%%%%%%%%%%%%%%%%%%%%%%%%%%%%%%%%%%%

%----------------------------------------------------------------------------------------
%	PACKAGES AND OTHER DOCUMENT CONFIGURATIONS
%----------------------------------------------------------------------------------------

\documentclass[letterpaper]{twentysecondcv} % a4paper for A4

% Command for printing skill overview bubbles
\newcommand\skills{ 
~
	\smartdiagram[bubble diagram]{
        \textbf{Data}\\\textbf{Engineering},
        \textbf{Full Stack}\\\textbf{Web Dev},
        \textbf{~~~~~OOP~~~~~~},
        \textbf{Mobile}\\\textbf{Dev},
        \textbf{Machine}\\\textbf{Learning},
        \textbf{Test}\\\textbf{~~Automation~~},
        \textbf{Statistical}\\\textbf{Analysis}
    }
}

% Programming skill bars
% changed \textbullet to \bullet
\programming{{$\bullet$ C/C++/ 5},  $\bullet$ R / 3}, {$\bullet$ \LaTeX / 3.5}, {HTML5 $\bullet$ JS $\bullet$ Python / 5}

% Projects text
\projects{
%% Locate transcript for both undergrad and graduate 

    \textbf{BIO544??} Created a 2-dimensional hydrophoic-polar (2D HP) protein folder utilizing an Ant Colony Optimization Algorithm in Python
    %\url{https://en.wikipedia.org/wiki/Hydrophobic-polar_protein_folding_model}
    %\url{https://en.wikipedia.org/wiki/Ant_colony_optimization_algorithms}
    \textbf{PHY477} Validated predicted reverberation times based on blue prints of the \href{Howard Performing Arts Center}{https://howard.andrews.edu}
    \textbf{PHY551} Josepheon junction

    \textbf{PHY515} Parallelized my dissertation utilizing ScaLAPACK
    %\url{http://www.netlib.org/scalapack/}
    \textbf{PHY516} Implemented a cellular automata traffic simulator in Python

    \textbf{Thesis}: Validation and Refinement of Coarse-Grained Protein Models
    
    \textbf{NSF REU} - Implemented a Monte Carlo modeler kinesin processivity

    \textbf{\href{PhysicsAMC}{https://github.com/jphafner/physicsAMC}} - A comprehensive test bank of physics questions

    A comprehensive physics question bank with an lpeg parser for question selection,
     that can generate professionally typeset exams:
    \href{exam}{https://github.com/jphafner/physicsAMC/blob/mc-project/kinematics-exam/sample-exam.pdf}
    \href{quiz1}{https://github.com/jphafner/physicsAMC/blob/mc-project/kinematics-full/sample-full.pdf}
    \href{quiz2}{https://github.com/jphafner/physicsAMC/blob/mc-project/kinematics-half/sample-half.pdf}
}

%----------------------------------------------------------------------------------------
%	 PERSONAL INFORMATION
%----------------------------------------------------------------------------------------
% If you don't need one or more of the below, just remove the content leaving the command, e.g. \cvnumberphone{}

\cvname{Jeffrey Hafner} % Your name
\cvjobtitle{Physicist} % title/career

\cvlinkedin{/in/jphafner}
\cvgithub{jphafner}
\cvnumberphone{(315) 532 0278} % Phone number
\cvsite{jphafner.github.io} % Personal website
\cvmail{jphafner@buffalo.edu} % Email address

%----------------------------------------------------------------------------------------

\begin{document}

\makeprofile % Print the sidebar

%----------------------------------------------------------------------------------------
%	 EDUCATION
%----------------------------------------------------------------------------------------
\section{Education}

\begin{twenty} % Environment for a list with descriptions
	\twentyitem
    	{2006 - 2012}
        {}
        {Pdh., Physics \textnormal{(GPA: 3.51)}}
        {\href{http://www.buffalo.edu/}{University at Buffalo, SUNY}}
        {}{}
	\twentyitem
    	{2001 - 2006}
		{}
        {MS/BS., Biophysics}
        {\href{http://www.andrews.edu/}{Andrews University, Michigan}}
        {}{}
	%\twentyitem{<dates>}{<title>}{<organization>}{<location>}{<description>}
\end{twenty}

\section{Research}
\begin{twenty}
	\twentyitem
    	{2008- 2011}
		{Buffalo., NY}
        {Phd. Candidate, Graduate Research Assistant}
        {\href{http://www.buffalo.edu/}{University at Buffalo, SUNY}}
        {}
        {
       	\textbf{Thesis}: Validation and Refinement of Coarse-Grained Protein Models
        {\begin{itemize}
            \item Validated coarse grained protein models against x-ray temperature factors using structural data from The \href{http://www.rcsb.org/pdb/home/home.do}{Protein Data Bank}
            \item \textbf{Tools}: C, BLAST, Python, matplotlib \vspace{2mm}
		\end{itemize}}
        }
\end{twenty}

\section{Publications}

Hafner J \& Zheng W. All-atom modeling of anisotropic atomic fluctuations in protein crystal structures. J. Chem. Phys. 135, 144114 (2011).
   \href{http://www.acsu.buffalo.edu/~wjzheng/Hafner_jcp2011.pdf}{PDF}
   \href{http://aip.scitation.org/doi/10.1063/1.3646312}{DOI}

Hafner J \& Zheng W. Optimal modeling of atomic fluctuations in protein crystal structures for weak crystal contact interactions. J. Chem. Phys. 132, 014111 (2010).
  \href{http://www.acsu.buffalo.edu/~wjzheng/Hafner_jcp2010.pdf}{PDF}
  \href{http://aip.scitation.org/doi/10.1063/1.3288503}{DOI}

Hafner J \& Zheng W. Approximate normal mode analysis based on vibrational subsystem analysis with high accuracy and efficiency. J. Chem. Phys. 130, 194111 (2009).
 \href{http://www.acsu.buffalo.edu/~wjzheng/Hafner_jcp2009.pdf}{PDF}
 \href{http://aip.scitation.org/doi/10.1063/1.3141022}{DOI}



%----------------------------------------------------------------------------------------
%	 EXPERIENCE
%----------------------------------------------------------------------------------------

\section{Experience}

\begin{twenty} % Environment for a list with descriptions
\twentyitem
    	{August 2016 -}
		{Present}
        {Unix/Automation Engineeer}
        {\href{http://www.ipsoft.com/}{IPsoft}}
        {New York, NY}
        {\begin{itemize}
        \item Provide Unix and automation services to clients through an ITIL framework
        \item Utilizing \href{https://ansible.com}{Ansible} and \href{https://www.ipsoft.com/ipcenter/}{IPsoft automatas} (proprietary, mostly javascript) %\href{IPsoft 1desk}{https://www.ipsoft.com/1desk/}
        \end{itemize}}
        \\
	\twentyitem
    	{August 2015 -}
		{May 2016}
        {Physics Teacher}
        {\href{http://www.mastersny.org/}{The Masters School}}
        {Dobbs Ferry, NY}
        {
        {\begin{itemize}
        \item Teaching 11\textsuperscript{th} grade and AP Physics B Mechanics.
        \item An example lesson plan \href{physicsReport}{https://github.com/jphafner/physicsReport}
          A lab report template and lesson plan that was utilized 
    	\end{itemize}}
        }
    \\   
    \twentyitem
   		{April 2014 -}
		{May 2015}
        {Physics Teacher}
        {\href{http://www.baltimorecityschools.org}{Baltimore City Public Schools}}
        {Baltimore, MD}
        {
        {\begin{itemize}
        \item Teaching \href{physics first}{https://en.wikipedia.org/wiki/Physics_First} at \href{https://mervo.org}{Mervo}.
        \item Implemented a unique assessment system utilizing lua, lpeg, \LaTeX and tikZ, that was later utilized at Masters.
        %\href{PhysicsAMC}{https://github.com/jphafner/physicsAMC}
    \end{itemize}}
        }
     \\
     \twentyitem
   		{Feb 2012 -}
		{Dec 2012}
        {Postdoctoral}
        {\href{http://www.ou.edu/cas/chemistry/}{Department of Chemistry \& Biochemistry, The University of Oklahoma}
        {\href{http://pharmacy.umaryland.edu}{University of Maryland School of Pharmacy}
        }
        {Norman, OK and Baltimore, MD}
        {
        \begin{itemize}
        \item Implementation of Particle Mesh Ewald electrostatics for Continous constant pH Molecular Dynamics in CHARMM
            %\url{https://www.charmm.org/charmm/program/}
            %\url{https://hpc.nih.gov/apps/charmm/c39b2html/phmd.html}
            %\url{https://sites.google.com/site/cphmdtutorial/objective}
    \end{itemize}
    	}
     \twentyitem
   		{Dec 2013 -}
		{Apr 2015}
        {Research Assistant}
        {\href{http://www.buffalo.edu/}{University at Buffalo, SUNY}}
        {Buffalo, NY}
        {
        \begin{itemize}
        \item Published three peer reviewed papers
        \item Implemented all code using C and Python in a \href{http://www.buffalo.edu/ccr.html}{High Performance Computing Environment}
    \end{itemize}
    	}
        
	%\twentyitem{<dates>}{<title>}{<location>}{<description>}
\end{twenty}

\end{document} 
