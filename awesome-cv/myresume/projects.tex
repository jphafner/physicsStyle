%-------------------------------------------------------------------------------
%	SECTION TITLE
%-------------------------------------------------------------------------------
\cvsection{Projects}


%-------------------------------------------------------------------------------
%	CONTENT
%-------------------------------------------------------------------------------
\begin{cventries}
  

%---------------------------------------------------------
%\subsection{Courses}

%---------------------------------------------------------
  \cventry
    {{\href{www.psu.edu}{Penn State University}}}
    {\href{www.nsf.gov}{NSF} REU}
    {University Park, PA}
    {Summer 2004}
    {Implemented a monte carlo modeler in Matlab to make falsifiable predictions concerning kinesin processivity under William Hancock.}

%---------------------------------------------------------
  \cventry
    {\href{www.andrews.edu}{Andrews University}}
    {PHY447: Advanced Lab 2}
    {Berrien Springs, MI} {Spring 2005} % Date
    {Validated predicted reverberation times based on blue prints of the \href{howard.andrews.edu}{Howard Performing arts center}}


%---------------------------------------------------------
  \cventry
    {\href{www.buffalo.edu}{University at Buffalo}}
    {PHY506: Computational Physics 2}
    {Buffalo, NY} {Spring 2008}
    {Implemented a cellular automata traffic modeler in \href{www.python.org}{Python} to investigate phase transitions in traffic}


%---------------------------------------------------------
  \cventry
    {\href{www.buffalo.edu}{University at Buffalo}}
	{PHY515: High Performance Computing 1}
    {Buffalo, NY} {Fall 2008}
    {Parallelized my dissertation utilizing ScaLAPACK.}

%---------------------------------------------------------
  \cventry
    {\href{www.buffalo.edu}{University at Buffalo}}
  	{PHY551: Grad Physics Laboratory 1}
    {Buffalo, NY} {Fall 2007}
    {
    \begin{cvitems}
        \item {Created \href{https://en.wikipedia.org/wiki/Josephson_effect}{Josephson junctions} for use in super conductive conditions}
        \item {Utilized a \href{https://en.wikipedia.org/wiki/Scanning_tunneling_microscope}{Scanning Tunneling Microscope} to investigate surface electron structure}
    \end{cvitems}
    }

%---------------------------------------------------------
  \cventry
    {\href{www.buffalo.edu}{University at Buffalo}}
    {CSE536: Computational Biology}
    {Buffalo, NY} {Fall 2011}
	{Implemented a 2D Hydrophobic-Hydrophilic Protein folder utilizing an \href{https://en.wikipedia.org/wiki/Ant_colony_optimization_algorithms}{Ant Colony Optimization Algorithm} in \href{www.python.org}{Python}.}

%---------------------------------------------------------
  \cventry
    {\href{www.buffalo.edu}{University at Buffalo}}
    {Doctoral Dissertation}
    {Buffalo, NY} {2008--2011}
{
\begin{cvitems}
    \item {titled: \emph{Validation and Refinement of Course Grained Protein Models}}
    \item {About a 100 pages of text, Over 5000 lines of C, and over 1000 lines of \href{www.python.org}{Python}.}
    \item {Work was performed on the computing resources of \href{ccr.buffalo.edu}{UB Center for Computational Research}}
\end{cvitems}
}

%\subsection{github}

%---------------------------------------------------------
  \cventry
    {Physics Teacher} % Job title
    {\href{https://github.com/jphafner/physicsAMC}{physicsAMC}} % Location
    {PhysicsAMC}
    {multiple locations}
    {2014--2016} % Date(s)
{
\begin{cvitems}
    \item {A comprehensive physics exam bank that utilizes an lpeg parser for question selection.}
    \item {This project enabled me to use an infinite redo policy on all assessments, without punishment, which was an important motivation for this project, and created some of my favorite memories.}
    \item {this project utilizes \LaTeX, \href{lua.org}{lua}, \href{http://www.inf.puc-rio.br/~roberto/lpeg/}{lpeg}, and \href{http://www.texample.net/tikz/}{tikz} for graphics, and contains more than a 100,000 lines of code.}
    \item { \href{https://github.com/jphafner/physicsAMC/blob/mc-project/kinematics-exam/sample-exam.pdf}{sample-exam}}
\end{cvitems}
}

%---------------------------------------------------------
  \cventry
    {Physics Teacher} % Job title
    {\href{https://github.com/jphafner/physicsReport}{physicsReport}}
    {2014}
    {An example lesson plan, and lab report template that I used while a physics teacher}

%---------------------------------------------------------
\end{cventries}

